% Options for packages loaded elsewhere
\PassOptionsToPackage{unicode}{hyperref}
\PassOptionsToPackage{hyphens}{url}
\PassOptionsToPackage{dvipsnames,svgnames,x11names}{xcolor}
%
\documentclass[
  letterpaper,
  DIV=11,
  numbers=noendperiod]{scrartcl}

\usepackage{amsmath,amssymb}
\usepackage{iftex}
\ifPDFTeX
  \usepackage[T1]{fontenc}
  \usepackage[utf8]{inputenc}
  \usepackage{textcomp} % provide euro and other symbols
\else % if luatex or xetex
  \usepackage{unicode-math}
  \defaultfontfeatures{Scale=MatchLowercase}
  \defaultfontfeatures[\rmfamily]{Ligatures=TeX,Scale=1}
\fi
\usepackage{lmodern}
\ifPDFTeX\else  
    % xetex/luatex font selection
\fi
% Use upquote if available, for straight quotes in verbatim environments
\IfFileExists{upquote.sty}{\usepackage{upquote}}{}
\IfFileExists{microtype.sty}{% use microtype if available
  \usepackage[]{microtype}
  \UseMicrotypeSet[protrusion]{basicmath} % disable protrusion for tt fonts
}{}
\makeatletter
\@ifundefined{KOMAClassName}{% if non-KOMA class
  \IfFileExists{parskip.sty}{%
    \usepackage{parskip}
  }{% else
    \setlength{\parindent}{0pt}
    \setlength{\parskip}{6pt plus 2pt minus 1pt}}
}{% if KOMA class
  \KOMAoptions{parskip=half}}
\makeatother
\usepackage{xcolor}
\setlength{\emergencystretch}{3em} % prevent overfull lines
\setcounter{secnumdepth}{-\maxdimen} % remove section numbering
% Make \paragraph and \subparagraph free-standing
\makeatletter
\ifx\paragraph\undefined\else
  \let\oldparagraph\paragraph
  \renewcommand{\paragraph}{
    \@ifstar
      \xxxParagraphStar
      \xxxParagraphNoStar
  }
  \newcommand{\xxxParagraphStar}[1]{\oldparagraph*{#1}\mbox{}}
  \newcommand{\xxxParagraphNoStar}[1]{\oldparagraph{#1}\mbox{}}
\fi
\ifx\subparagraph\undefined\else
  \let\oldsubparagraph\subparagraph
  \renewcommand{\subparagraph}{
    \@ifstar
      \xxxSubParagraphStar
      \xxxSubParagraphNoStar
  }
  \newcommand{\xxxSubParagraphStar}[1]{\oldsubparagraph*{#1}\mbox{}}
  \newcommand{\xxxSubParagraphNoStar}[1]{\oldsubparagraph{#1}\mbox{}}
\fi
\makeatother

\usepackage{color}
\usepackage{fancyvrb}
\newcommand{\VerbBar}{|}
\newcommand{\VERB}{\Verb[commandchars=\\\{\}]}
\DefineVerbatimEnvironment{Highlighting}{Verbatim}{commandchars=\\\{\}}
% Add ',fontsize=\small' for more characters per line
\usepackage{framed}
\definecolor{shadecolor}{RGB}{241,243,245}
\newenvironment{Shaded}{\begin{snugshade}}{\end{snugshade}}
\newcommand{\AlertTok}[1]{\textcolor[rgb]{0.68,0.00,0.00}{#1}}
\newcommand{\AnnotationTok}[1]{\textcolor[rgb]{0.37,0.37,0.37}{#1}}
\newcommand{\AttributeTok}[1]{\textcolor[rgb]{0.40,0.45,0.13}{#1}}
\newcommand{\BaseNTok}[1]{\textcolor[rgb]{0.68,0.00,0.00}{#1}}
\newcommand{\BuiltInTok}[1]{\textcolor[rgb]{0.00,0.23,0.31}{#1}}
\newcommand{\CharTok}[1]{\textcolor[rgb]{0.13,0.47,0.30}{#1}}
\newcommand{\CommentTok}[1]{\textcolor[rgb]{0.37,0.37,0.37}{#1}}
\newcommand{\CommentVarTok}[1]{\textcolor[rgb]{0.37,0.37,0.37}{\textit{#1}}}
\newcommand{\ConstantTok}[1]{\textcolor[rgb]{0.56,0.35,0.01}{#1}}
\newcommand{\ControlFlowTok}[1]{\textcolor[rgb]{0.00,0.23,0.31}{\textbf{#1}}}
\newcommand{\DataTypeTok}[1]{\textcolor[rgb]{0.68,0.00,0.00}{#1}}
\newcommand{\DecValTok}[1]{\textcolor[rgb]{0.68,0.00,0.00}{#1}}
\newcommand{\DocumentationTok}[1]{\textcolor[rgb]{0.37,0.37,0.37}{\textit{#1}}}
\newcommand{\ErrorTok}[1]{\textcolor[rgb]{0.68,0.00,0.00}{#1}}
\newcommand{\ExtensionTok}[1]{\textcolor[rgb]{0.00,0.23,0.31}{#1}}
\newcommand{\FloatTok}[1]{\textcolor[rgb]{0.68,0.00,0.00}{#1}}
\newcommand{\FunctionTok}[1]{\textcolor[rgb]{0.28,0.35,0.67}{#1}}
\newcommand{\ImportTok}[1]{\textcolor[rgb]{0.00,0.46,0.62}{#1}}
\newcommand{\InformationTok}[1]{\textcolor[rgb]{0.37,0.37,0.37}{#1}}
\newcommand{\KeywordTok}[1]{\textcolor[rgb]{0.00,0.23,0.31}{\textbf{#1}}}
\newcommand{\NormalTok}[1]{\textcolor[rgb]{0.00,0.23,0.31}{#1}}
\newcommand{\OperatorTok}[1]{\textcolor[rgb]{0.37,0.37,0.37}{#1}}
\newcommand{\OtherTok}[1]{\textcolor[rgb]{0.00,0.23,0.31}{#1}}
\newcommand{\PreprocessorTok}[1]{\textcolor[rgb]{0.68,0.00,0.00}{#1}}
\newcommand{\RegionMarkerTok}[1]{\textcolor[rgb]{0.00,0.23,0.31}{#1}}
\newcommand{\SpecialCharTok}[1]{\textcolor[rgb]{0.37,0.37,0.37}{#1}}
\newcommand{\SpecialStringTok}[1]{\textcolor[rgb]{0.13,0.47,0.30}{#1}}
\newcommand{\StringTok}[1]{\textcolor[rgb]{0.13,0.47,0.30}{#1}}
\newcommand{\VariableTok}[1]{\textcolor[rgb]{0.07,0.07,0.07}{#1}}
\newcommand{\VerbatimStringTok}[1]{\textcolor[rgb]{0.13,0.47,0.30}{#1}}
\newcommand{\WarningTok}[1]{\textcolor[rgb]{0.37,0.37,0.37}{\textit{#1}}}

\providecommand{\tightlist}{%
  \setlength{\itemsep}{0pt}\setlength{\parskip}{0pt}}\usepackage{longtable,booktabs,array}
\usepackage{calc} % for calculating minipage widths
% Correct order of tables after \paragraph or \subparagraph
\usepackage{etoolbox}
\makeatletter
\patchcmd\longtable{\par}{\if@noskipsec\mbox{}\fi\par}{}{}
\makeatother
% Allow footnotes in longtable head/foot
\IfFileExists{footnotehyper.sty}{\usepackage{footnotehyper}}{\usepackage{footnote}}
\makesavenoteenv{longtable}
\usepackage{graphicx}
\makeatletter
\newsavebox\pandoc@box
\newcommand*\pandocbounded[1]{% scales image to fit in text height/width
  \sbox\pandoc@box{#1}%
  \Gscale@div\@tempa{\textheight}{\dimexpr\ht\pandoc@box+\dp\pandoc@box\relax}%
  \Gscale@div\@tempb{\linewidth}{\wd\pandoc@box}%
  \ifdim\@tempb\p@<\@tempa\p@\let\@tempa\@tempb\fi% select the smaller of both
  \ifdim\@tempa\p@<\p@\scalebox{\@tempa}{\usebox\pandoc@box}%
  \else\usebox{\pandoc@box}%
  \fi%
}
% Set default figure placement to htbp
\def\fps@figure{htbp}
\makeatother

\KOMAoption{captions}{tableheading}
\makeatletter
\@ifpackageloaded{caption}{}{\usepackage{caption}}
\AtBeginDocument{%
\ifdefined\contentsname
  \renewcommand*\contentsname{Tabla de contenidos}
\else
  \newcommand\contentsname{Tabla de contenidos}
\fi
\ifdefined\listfigurename
  \renewcommand*\listfigurename{Listado de Figuras}
\else
  \newcommand\listfigurename{Listado de Figuras}
\fi
\ifdefined\listtablename
  \renewcommand*\listtablename{Listado de Tablas}
\else
  \newcommand\listtablename{Listado de Tablas}
\fi
\ifdefined\figurename
  \renewcommand*\figurename{Figura}
\else
  \newcommand\figurename{Figura}
\fi
\ifdefined\tablename
  \renewcommand*\tablename{Tabla}
\else
  \newcommand\tablename{Tabla}
\fi
}
\@ifpackageloaded{float}{}{\usepackage{float}}
\floatstyle{ruled}
\@ifundefined{c@chapter}{\newfloat{codelisting}{h}{lop}}{\newfloat{codelisting}{h}{lop}[chapter]}
\floatname{codelisting}{Listado}
\newcommand*\listoflistings{\listof{codelisting}{Listado de Listados}}
\makeatother
\makeatletter
\makeatother
\makeatletter
\@ifpackageloaded{caption}{}{\usepackage{caption}}
\@ifpackageloaded{subcaption}{}{\usepackage{subcaption}}
\makeatother
\makeatletter
\definecolor{QuartoInternalColor3}{rgb}{0.00,0.64,0.31}
\definecolor{QuartoInternalColor1}{rgb}{0.91,0.36,0.35}
\definecolor{QuartoInternalColor4}{rgb}{0.00,0.40,0.00}
\definecolor{QuartoInternalColor7}{rgb}{0.00,0.45,0.15}
\definecolor{QuartoInternalColor8}{rgb}{0.60,0.00,0.00}
\definecolor{QuartoInternalColor6}{rgb}{0.38,0.38,0.38}
\definecolor{QuartoInternalColor2}{rgb}{0,0,0}
\definecolor{QuartoInternalColor5}{rgb}{0.00,0.00,1.00}
\makeatother

\ifLuaTeX
\usepackage[bidi=basic]{babel}
\else
\usepackage[bidi=default]{babel}
\fi
\babelprovide[main,import]{spanish}
% get rid of language-specific shorthands (see #6817):
\let\LanguageShortHands\languageshorthands
\def\languageshorthands#1{}
\usepackage{bookmark}

\IfFileExists{xurl.sty}{\usepackage{xurl}}{} % add URL line breaks if available
\urlstyle{same} % disable monospaced font for URLs
\hypersetup{
  pdftitle={{[}Actividad extracurricular 12{]} web scraping},
  pdfauthor={David Pilataxi},
  pdflang={es},
  colorlinks=true,
  linkcolor={blue},
  filecolor={Maroon},
  citecolor={Blue},
  urlcolor={Blue},
  pdfcreator={LaTeX via pandoc}}


\title{{[}Actividad extracurricular 12{]} web scraping}
\author{David Pilataxi}
\date{2025-08-01}

\begin{document}
\maketitle

\renewcommand*\contentsname{Tabla de Contenidos}
{
\hypersetup{linkcolor=}
\setcounter{tocdepth}{3}
\tableofcontents
}

\section{Web Scrapping}\label{web-scrapping}

\begin{itemize}
\tightlist
\item
  David Pilataxi
\item
  Gr1cc
\item
  8 de enero de 2025
\end{itemize}

\subsection{Enlace al Repositorio}\label{enlace-al-repositorio}

LINK:
https://github.com/DavidPilataxi/MetodosNumericosGr1cc/blob/main/Actividad\%20Extra\%2012\%20Web\%20Scrapping/WebScraping.ipynb

\section{1. Objetivos}\label{objetivos}

\begin{itemize}
\tightlist
\item
  Revisar qué es web scraping
\item
  Realizar una prueba en python para dos librerías diferentes
\item
  Realizar scraping de un sitio web de su elección
\end{itemize}

\section{2. Introducción}\label{introducciuxf3n}

Es el proceso de extraer datos de sitios web de manera automatizada
utilizando herramientas o scripts. Esto se logra accediendo al contenido
de las páginas web, ya sea analizando el HTML, extrayendo datos
estructurados, o interactuando con elementos dinámicos.

Aplicaciones comunes de web scraping:

\begin{itemize}
\tightlist
\item
  Análisis de precios en sitios de comercio electrónico.
\item
  Monitorización de noticias o contenidos en tiempo real.
\item
  Recolección de información para proyectos de investigación o estudios
  de mercado.
\item
  Extracción de datos de directorios en línea o bases de datos
  accesibles públicamente.
\end{itemize}

Herramientas populares para web scraping:

\begin{itemize}
\tightlist
\item
  BeautifulSoup (Python) para extraer datos de HTML.
\item
  Selenium para interactuar con páginas dinámicas.
\item
  Scrapy, un framework avanzado para tareas de scraping masivo.
\end{itemize}

\section{3. Procedimiento}\label{procedimiento}

Web Scraping: BeautifulSoup vs Scrapy

\textbf{Configuración inicial}

\begin{itemize}
\tightlist
\item
  !pip install requests beautifulsoup4 pandas scrapy
\end{itemize}

\subsubsection{3.1. Web Scraping con BeautifulSoup.
-}\label{web-scraping-con-beautifulsoup.--}

\begin{Shaded}
\begin{Highlighting}[]
\ImportTok{import}\NormalTok{ requests}
\ImportTok{from}\NormalTok{ bs4 }\ImportTok{import}\NormalTok{ BeautifulSoup}
\ImportTok{import}\NormalTok{ pandas }\ImportTok{as}\NormalTok{ pd}

\CommentTok{\# URL del sitio web}
\NormalTok{url }\OperatorTok{=} \StringTok{"https://www.python.org/blogs/"}

\CommentTok{\# Enviar la solicitud HTTP}
\NormalTok{response }\OperatorTok{=}\NormalTok{ requests.get(url)}

\CommentTok{\# Verificar si la solicitud fue exitosa}
\ControlFlowTok{if}\NormalTok{ response.status\_code }\OperatorTok{==} \DecValTok{200}\NormalTok{:}
\NormalTok{    soup }\OperatorTok{=}\NormalTok{ BeautifulSoup(response.text, }\StringTok{"html.parser"}\NormalTok{)}

    \CommentTok{\# Ajustar los selectores para títulos y enlaces}
\NormalTok{    titles }\OperatorTok{=}\NormalTok{ [title.text.strip() }\ControlFlowTok{for}\NormalTok{ title }\KeywordTok{in}\NormalTok{ soup.select(}\StringTok{"li h3"}\NormalTok{)]}
\NormalTok{    links }\OperatorTok{=}\NormalTok{ [link[}\StringTok{\textquotesingle{}href\textquotesingle{}}\NormalTok{] }\ControlFlowTok{for}\NormalTok{ link }\KeywordTok{in}\NormalTok{ soup.select(}\StringTok{"li h3 a"}\NormalTok{)]}

    \CommentTok{\# Crear DataFrame}
\NormalTok{    data }\OperatorTok{=}\NormalTok{ pd.DataFrame(\{}\StringTok{"Título"}\NormalTok{: titles, }\StringTok{"Enlace"}\NormalTok{: links\})}
\NormalTok{    data.to\_csv(}\StringTok{"python\_org\_blogs.csv"}\NormalTok{, index}\OperatorTok{=}\VariableTok{False}\NormalTok{)}

    \BuiltInTok{print}\NormalTok{(}\StringTok{"Datos extraídos de Python.org:"}\NormalTok{)}
    \BuiltInTok{print}\NormalTok{(data.head())}
\ControlFlowTok{else}\NormalTok{:}
    \BuiltInTok{print}\NormalTok{(}\SpecialStringTok{f"Error: No se pudo acceder al sitio web. Código de estado }\SpecialCharTok{\{}\NormalTok{response}\SpecialCharTok{.}\NormalTok{status\_code}\SpecialCharTok{\}}\SpecialStringTok{"}\NormalTok{)}
\end{Highlighting}
\end{Shaded}

\begin{verbatim}
Datos extraídos de Python.org:
                                              Título  \
0     PSF Grants: Program & Charter Updates (Part 1)   
1       PSF Grants: Program & Charter Updates (TLDR)   
2     PSF Grants: Program & Charter Updates (Part 3)   
3  Announcing Python Software Foundation Fellow M...   
4                       Python 3.14.0 alpha 3 is out   

                                              Enlace  
0  https://pyfound.blogspot.com/2024/12/psf-grant...  
1  https://pyfound.blogspot.com/2024/12/psf-grant...  
2  https://pyfound.blogspot.com/2024/12/12psf-gra...  
3  https://pyfound.blogspot.com/2024/12/announcin...  
4  https://pythoninsider.blogspot.com/2024/12/pyt...  
\end{verbatim}

\pandocbounded{\includegraphics[keepaspectratio]{Imagen1.png}}

\subsubsection{3.2. Web Scraping con
Scrapy}\label{web-scraping-con-scrapy}

Pasos en terminal: - pip install scrapy - scrapy startproject
quotes\_scraper - scrapy crawl quotes -o quotes.json\\
El siguiente es el código para el web scrapping, y acontinuación, se
presenta el resultado tras correrlo desde la terminal en un archivo
independiente.

\begin{Shaded}
\begin{Highlighting}[]

\KeywordTok{class}\NormalTok{ QuotesSpider(scrapy.Spider):}
\NormalTok{    name }\OperatorTok{=} \StringTok{\textquotesingle{}quotes\textquotesingle{}}
\NormalTok{    start\_urls }\OperatorTok{=}\NormalTok{ [}\StringTok{\textquotesingle{}http://quotes.toscrape.com/\textquotesingle{}}\NormalTok{]}

    \KeywordTok{def}\NormalTok{ parse(}\VariableTok{self}\NormalTok{, response):}
        \CommentTok{\# Extraer todas las citas de la página}
        \ControlFlowTok{for}\NormalTok{ quote }\KeywordTok{in}\NormalTok{ response.css(}\StringTok{\textquotesingle{}div.quote\textquotesingle{}}\NormalTok{):}
            \ControlFlowTok{yield}\NormalTok{ \{}
                \StringTok{\textquotesingle{}text\textquotesingle{}}\NormalTok{: quote.css(}\StringTok{\textquotesingle{}span.text::text\textquotesingle{}}\NormalTok{).get(),}
                \StringTok{\textquotesingle{}author\textquotesingle{}}\NormalTok{: quote.css(}\StringTok{\textquotesingle{}span small::text\textquotesingle{}}\NormalTok{).get(),}
                \StringTok{\textquotesingle{}tags\textquotesingle{}}\NormalTok{: quote.css(}\StringTok{\textquotesingle{}div.tags a.tag::text\textquotesingle{}}\NormalTok{).getall(),}
\NormalTok{            \}}

        \CommentTok{\# Seguir al siguiente página, si existe}
\NormalTok{        next\_page }\OperatorTok{=}\NormalTok{ response.css(}\StringTok{\textquotesingle{}li.next a::attr(href)\textquotesingle{}}\NormalTok{).get()}
        \ControlFlowTok{if}\NormalTok{ next\_page:}
            \ControlFlowTok{yield}\NormalTok{ response.follow(next\_page, }\VariableTok{self}\NormalTok{.parse)}
\end{Highlighting}
\end{Shaded}

\begin{Highlighting}
\textcolor{black}{NameError: name 'scrapy' is not defined}
\textcolor{black}{}\textcolor{QuartoInternalColor1}{---------------------------------------------------------------------------}\textcolor{QuartoInternalColor2}{}
\textcolor{QuartoInternalColor2}{}\textcolor{QuartoInternalColor1}{NameError}\textcolor{QuartoInternalColor2}{                                 Traceback (most recent call last)}
\textcolor{QuartoInternalColor2}{Cell }\textcolor{QuartoInternalColor3}{In[4], line 1}\textcolor{QuartoInternalColor2}{}
\textcolor{QuartoInternalColor2}{}\textcolor{QuartoInternalColor3}{----> 1}\textcolor{QuartoInternalColor2}{ }\textcolor{QuartoInternalColor4}{class}\textcolor{QuartoInternalColor2}{ }\textcolor{QuartoInternalColor5}{QuotesSpider}\textcolor{QuartoInternalColor2}{(scrapy}\textcolor{QuartoInternalColor6}{.}\textcolor{QuartoInternalColor2}{Spider):}
\textcolor{QuartoInternalColor2}{}\textcolor{QuartoInternalColor7}{      2}\textcolor{QuartoInternalColor2}{     name }\textcolor{QuartoInternalColor6}{=}\textcolor{QuartoInternalColor2}{ }\textcolor{QuartoInternalColor8}{'}\textcolor{QuartoInternalColor2}{}\textcolor{QuartoInternalColor8}{quotes}\textcolor{QuartoInternalColor2}{}\textcolor{QuartoInternalColor8}{'}\textcolor{QuartoInternalColor2}{}
\textcolor{QuartoInternalColor2}{}\textcolor{QuartoInternalColor7}{      3}\textcolor{QuartoInternalColor2}{     start_urls }\textcolor{QuartoInternalColor6}{=}\textcolor{QuartoInternalColor2}{ [}\textcolor{QuartoInternalColor8}{'}\textcolor{QuartoInternalColor2}{}\textcolor{QuartoInternalColor8}{http://quotes.toscrape.com/}\textcolor{QuartoInternalColor2}{}\textcolor{QuartoInternalColor8}{'}\textcolor{QuartoInternalColor2}{]}
\textcolor{QuartoInternalColor2}{}\textcolor{QuartoInternalColor1}{NameError}\textcolor{QuartoInternalColor2}{: name 'scrapy' is not defined}
\end{Highlighting}

\pandocbounded{\includegraphics[keepaspectratio]{Imagen2.png}}

\section{4. Conclusiones}\label{conclusiones}

\begin{itemize}
\item
  BeautifulSoup destaca por su simplicidad, lo que lo convierte en una
  opción ideal para proyectos pequeños o tareas de scraping sencillas.
\item
  Scrapy, por otro lado, es una herramienta escalable, diseñada para
  manejar proyectos más complejos o grandes volúmenes de datos, lo que
  lo convierte en la opción preferida para scraping masivo.
\item
  Entre las limitaciones comunes en el scraping, se encuentran los
  problemas legales, ya que es importante respetar las políticas de uso
  de los sitios web. Además, pueden surgir bloqueos o restricciones en
  los sitios que dificultan la recolección de datos.
\item
  Una consideración adicional es la gestión de la eficiencia, ya que
  Scrapy, al ser más robusto, permite manejar peticiones concurrentes y
  tiempos de espera, mientras que BeautifulSoup puede volverse menos
  eficiente en proyectos más grandes debido a su naturaleza más simple.
\end{itemize}

\section{5. Referencias
Bibliográficas}\label{referencias-bibliogruxe1ficas}

\begin{itemize}
\tightlist
\item
  \textbf{Requests:}
\end{itemize}

Reitz, K. (2023). Requests: HTTP for Humans. Python Software Foundation.
Recuperado de https://docs.python-requests.org/en/latest/

\begin{itemize}
\tightlist
\item
  \textbf{BeautifulSoup:}
\end{itemize}

Richardson, L. (2023). Beautiful Soup Documentation. Python Software
Foundation. Recuperado de https://www.crummy.com/software/BeautifulSoup/

\begin{itemize}
\tightlist
\item
  \textbf{Pandas:}
\end{itemize}

McKinney, W. (2010). Data Structures for Statistical Computing in
Python. In Proceedings of the 9th Python in Science Conference
(pp.~51-56). Recuperado de https://pandas.pydata.org/

\begin{itemize}
\tightlist
\item
  \textbf{Scrapy:}
\end{itemize}

Scrapy Developers. (2023). Scrapy Documentation. Recuperado de
https://docs.scrapy.org/en/latest/

\begin{itemize}
\tightlist
\item
  \textbf{Sitio web objetivo:}
\end{itemize}

Quotes to Scrape. (s.f.). Recuperado de https://quotes.toscrape.com




\end{document}
